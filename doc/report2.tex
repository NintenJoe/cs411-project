%	@file FinalDocumentation.tex
%	@author Thomas Bogue, Joseph Ciurej
%	@date Spring 2014
%	
%   LaTeX source file for the design document for the CS411 project.  More 
%   information about the requirements for this documentation file can be found
%   here: "https://wiki.cites.illinois.edu/wiki/display/cs411sp14/Final+Report+Instructions"

\documentclass{article}
\usepackage[parfill]{parskip}		% Package that improves paragraph formatting
\usepackage[pdftex]{graphicx}		% Package that faciliates figure insertion
\usepackage{float}					% Package that allows for arbitrary figure insertion
\usepackage{hyperref}				% Package that allows for text hyperlinking
\usepackage{tikz}					% Package that facilitates box rendering

% The following code block changes a few parameters of the page to make the
% text fill up more space on any given page.  For our liking, the default 'article'
% template is too sparse.
\addtolength{\voffset}{-2cm}
\addtolength{\textheight}{4cm}
\addtolength{\oddsidemargin}{-1.5cm}
\addtolength{\textwidth}{2cm}

% This command creates blue boxes around class names.
% @see "http://tex.stackexchange.com/questions/36401/drawing-boxes-around-words"
\newcommand\appname[1][]{\tikz[overlay]\node[fill=green!20,inner sep=2pt, anchor=text, rectangle, rounded corners=1mm,#1] { DatBigCuke };\phantom{ DatBigCuke}}
\newcommand\gcalendar[1][]{\href{https://www.google.com/calendar}{Google Calendar} }
\newcommand\gravatar[1][]{\href{http://en.gravatar.com/}{Gravatar} }

\newcommand{\insertdiagram}[2]
{
	\begin{figure}[H]
		\centering
		\fbox{\includegraphics[height=#2]{figures/#1}}
		\caption{UML Diagram for the #1 Class}
	\end{figure}
}

\begin{document}

	% Title Page %
	\title{DatBigCuke: Final Project Report}
	\author{Eunsoo Roh, Joshua Halstead, Kyle Nusbaum, Thomas Bogue, Joseph Ciurej \\
		University of Illinois: Urbana-Champaign (CS411)}
	\date{\today}
	\maketitle

	% Table of Contents %
	\tableofcontents
	\clearpage

	% Body %
	\section[Project Overview]{Brief Project Overview}
	The primary purpose of this project was to create an online service for college
	students to help them manage their academic deadlines.  Our primary hope
	is that the application can serve a centralized repository for college
	students' deadlines and aid them in organizing these deadlines in a 
	straightforward and digestable manner.  
	
	The application is meant to be used by many students in college courses to 
	facilitate the crowdsourcing of deadline times for course homeworks and other 
	various assignments.  Additionally, the application provides functionality to allow 
	student groups to organize group meetings to work towards completing their related deadlines.

		\subsection[Motiviation]{Project Motivation}
		The main motivation for creating this service was a common need among
		the group members for a centralized deadline management system.  We all
		found that the highly variate means by which courses and college groups
		distribute deadlines made effective deadline management a near impossibility.
		In creating the \appname web service, we aim to eliminate this problem by 
		allowing users to congregate and collaborate on deadline data entry.  
		
		Additionally, we've found that organizing group meetings with the purpose of 
		working towards completing deadlines is an equally difficult task with the
		large variability in group member schedules.  The need to efficiently find
		potential times for such meetings was another primary motivator for the
		features we implemented in the \appname service.

		\subsection[Affordances]{Project Affordances}
		The \appname service has the capacity to help users accomplish a number of
		tasks, including more efficient deadline management, more effective
		group scheduling, and more centralized group organization.
		
		Users can sign up for groups in \appname and enter deadlines associated 
		with these groups. Once enough users of a group have entered a deadline
		for a particular assignment, the official time for this deadline is
		determined by aggregating these user-entered deadlines using statisical
		analysis methods.

		The integration of autocomplete makes the process of signing up for
		academic groups seamless.  After seamlessly signing up for their
		courses through the \appname interface, users can begin to create their
		own user subgroups and begin scheduling meetings to accomplish the
		tasks associated with their deadlines.  
		
		The means of finding appropriate scheduling times is made painless 
		through integration with \gcalendar, which the application pulls scheduling 
		data from to determine the availability times of each group member.  
		The \appname service uses each users selected \gcalendar calendars in
		addition to a few other user input values (e.g. off limits times, deadline 
		for the meeting, et cetera) in order to determine optimal meeting times.  
		Also, through the use of asynchronous email handlers, the \appname service 
		allows users to quickly and easily propogate calculated meeting times, 
		making the process of scheduling even easier.

	\section[Functions and Features]{Project Functions and Features}
	In order to more fully describe the affordances of the \appname application,
	we'll now flesh out all the functions and features of the service in more
	detail.  Specifically, we'll detail the primary features of the application 
	in brief, then move to overview a few of the basic functions provided by the 
	application before finally highlighting the two key features of the application
	(i.e. deadline aggregation and group meeting scheduling with \gcalendar).

		\subsection[Feature List]{Concise Functions and Features List}
		\begin{itemize}
			\item Users can sign up for the service by providing their email 
			address and credentials.  Each user must verify their email address 
			before being allowed to utilize the service.

			\item Authenticated users may view and edit their personal information 
			on their profile page (including the \gravatar icons associated with their
			sign-in email).

			\item Users can link their account to \gcalendar from their main
			profile page by filling out the linked Google service verification form.

			\item Authenticated users can view and modify the hierarchy of academic
			groups that they're signed up for on the groups page.

			\item Users can add courses through the root group page (i.e. the
			user institution group page) with user entry assisted by autocomplete.
			Users can also add private subgroups to courses through one of the groups page.

			\item Users can add additional members to private subgroups with
			user email entry assisted by autocomplete.

			\item Users can freely leave groups that aren't dependent on their
			presence.

			\item Users can add deadlines for assignments associated with groups
			through the group interface with assignment name entry assisted by
			autocomplete.

			\item Once enough users have added a deadline for an assignment,
			these deadlines are aggregated using statistical analysis methods
			and broadcast to all users (and listed as community-driven deadlines).

			\item Users can request the scheduling of a group meeting for a
			particular deadline through the group interface.  This interface 
			allows the user to customize the meeting duration, off-limits
			times, and the required attendees.  A list of good meeting times
			will be returned based on this information and user \gcalendar data.
			
			\item Given a list of potential meeting times, the users are
			provided an interface that allows them to choose a good meeting
			time and to broadcast this time with a custom message to all group
			members.
		\end{itemize}

		\subsection[Basic Functions]{Overview of A Few Basic Functions}
		The core of the Jenkins chat plugin consists of the types used to
		implement the server and client components of the client-server model.
		For each of these primary components, the listing below describes the
		major functionality encapsulated by that primary component and provides
		a visual aid for its role in the project through a 
		\href{http://www.csci.csusb.edu/dick/samples/uml0.html}{UML diagram}.

		\subsection[Advanced Functions]{Overview of Advanced Features}
		The core of the Jenkins chat plugin consists of the types used to
		implement the server and client components of the client-server model.
		For each of these primary components, the listing below describes the
		major functionality encapsulated by that primary component and provides
		a visual aid for its role in the project through a 
		\href{http://www.csci.csusb.edu/dick/samples/uml0.html}{UML diagram}.

	\section[Project Implementation]{Project Implementation and Design}
	The core of the Jenkins chat plugin consists of the types used to
	implement the server and client components of the client-server model.
	For each of these primary components, the listing below describes the
	major functionality encapsulated by that primary component and provides
	a visual aid for its role in the project through a 
	\href{http://www.csci.csusb.edu/dick/samples/uml0.html}{UML diagram}.

		\subsection[Database Design]{Data and Database Design}
		The core of the Jenkins chat plugin consists of the types used to
		implement the server and client components of the client-server model.
		For each of these primary components, the listing below describes the
		major functionality encapsulated by that primary component and provides
		a visual aid for its role in the project through a 
		\href{http://www.csci.csusb.edu/dick/samples/uml0.html}{UML diagram}.

		\subsection[Implementation Schedule]{Project Implementation Schedule and Tasks}
		The core of the Jenkins chat plugin consists of the types used to
		implement the server and client components of the client-server model.
		For each of these primary components, the listing below describes the
		major functionality encapsulated by that primary component and provides
		a visual aid for its role in the project through a 
		\href{http://www.csci.csusb.edu/dick/samples/uml0.html}{UML diagram}.

			\subsubsection[Task Delegation]{Delegation of Project Tasks}
			The core of the Jenkins chat plugin consists of the types used to
			implement the server and client components of the client-server model.
			For each of these primary components, the listing below describes the
			major functionality encapsulated by that primary component and provides
			a visual aid for its role in the project through a 
			\href{http://www.csci.csusb.edu/dick/samples/uml0.html}{UML diagram}.

			\subsubsection[Challenges]{Technical Challenges}
			The core of the Jenkins chat plugin consists of the types used to
			implement the server and client components of the client-server model.
			For each of these primary components, the listing below describes the
			major functionality encapsulated by that primary component and provides
			a visual aid for its role in the project through a 
			\href{http://www.csci.csusb.edu/dick/samples/uml0.html}{UML diagram}.


	\section[Appendix]{Miscellaneous Project Information}
	The core of the Jenkins chat plugin consists of the types used to
	implement the server and client components of the client-server model.
	For each of these primary components, the listing below describes the
	major functionality encapsulated by that primary component and provides
	a visual aid for its role in the project through a 
	\href{http://www.csci.csusb.edu/dick/samples/uml0.html}{UML diagram}.

		\subsection[Dependencies]{Sources and Dependencies}
		Installation for the chat plugin has the following dependencies, which
		must be installed before the plugin itself can be installed:

		\begin{description}
			\item[\href{http://www.oracle.com/technetwork/java/javase/downloads/jdk7-downloads-1880260.html}{Java Development Kit}] (v1.7+) \hfill \\
			A version of the standard Java development kit at least as recent 
			as version 1.7 is needed primarily to import extended networking 
			functionality only in later JDK versions.

			\item[\href{http://jenkins-ci.org/}{Jenkins}] (v1.509.3+) \hfill \\
			A version of Jenkins more recent than version 1.509.3 is needed
			to support the chat server plugin (older versions may work, but they
			have not been tested).

			\item[\href{https://github.com/mrniko/netty-socketio/releases}{Netty-SocketIO}] (v1.5.2+) \hfill \\ 
			An open source Java library that facilitates network communication 
			via websockets in Java.  This library is used by the chat plugin
			to aid in the transferral of mesages to and from chat clients.
		\end{description}

		After these dependencies are installed, installing the chat plugin itself
		is relatively straightforward.  The steps for performing this installation
		are as follows:

		\begin{enumerate}
			\item Navigate to the \textbf{plugin} subdirectory from the chat
			plugin source base directory.

			\item Run the script within this directory by the name of 
			\textbf{export.sh}.
		\end{enumerate}

		After performing the second step, a sequence of installations should
		follow.  After all these subsequent installations are complete, the
		plugin should be fully installed and accessible through Jenkins.

		\subsection[Other Information]{Other Relevant Information}
		TODO

\end{document}
