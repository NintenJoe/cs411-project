%	@file FinalDocumentation.tex
%	@author Thomas Bogue, Joseph Ciurej
%	@date Spring 2014
%	
%   LaTeX source file for the design document for the CS411 project.  More 
%   information about the requirements for this documentation file can be found
%   here: "https://wiki.cites.illinois.edu/wiki/display/cs411sp14/Final+Report+Instructions"

\documentclass{article}
\usepackage[parfill]{parskip}		% Package that improves paragraph formatting
\usepackage[pdftex]{graphicx}		% Package that faciliates figure insertion
\usepackage{float}					% Package that allows for arbitrary figure insertion
\usepackage{hyperref}				% Package that allows for text hyperlinking
\usepackage{tikz}					% Package that facilitates box rendering

% The following code block changes a few parameters of the page to make the
% text fill up more space on any given page.  For our liking, the default 'article'
% template is too sparse.
\addtolength{\voffset}{-2cm}
\addtolength{\textheight}{4cm}
\addtolength{\oddsidemargin}{-1.5cm}
\addtolength{\textwidth}{2cm}

% This command creates blue boxes around class names.
% @see "http://tex.stackexchange.com/questions/36401/drawing-boxes-around-words"
%\newcommand\classname[2][]{\tikz[overlay]\node[fill=blue!20,inner sep=2pt, anchor=text, rectangle, rounded corners=1mm,#1] {#2};\phantom{#2}}
%\newcommand\methodname[2][]{\tikz[overlay]\node[fill=green!20,inner sep=2pt, anchor=text, rectangle, rounded corners=1mm,#1] {#2};\phantom{#2}}
\newcommand{\classname}[1] {\texttt{#1}}
\newcommand{\methodname}[1] {\texttt{#1}}
\newcommand{\insertdiagram}[2]
{
	\begin{figure}[H]
		\centering
		\fbox{\includegraphics[height=#2]{figures/#1}}
		\caption{UML Diagram for the #1 Class}
	\end{figure}
}

\begin{document}

	% Title Page %
	\title{DatBigCuke: Final Project Report}
	\author{Eunsoo Roh, Joshua Halstead, Kyle Nusbaum, Thomas Bogue, Joseph Ciurej \\
		University of Illinois: Urbana-Champaign (CS411)}
	\date{\today}
	\maketitle

	% Table of Contents %
	\tableofcontents
	\clearpage

	% Body %
	\section[Project Overview]{Brief Project Overview}
	The purpose of this project was to create an online service for college
	students to help them schedule and manage their academic deadlines

	The purpose of this project is to create an extension to the continuous 
	integration server Jenkins that facilitates chat communication between
	users of the server.  The main feature of this extension is an augmentation
	to the standard Jenkins user web pages that allows for messages to be 
	transferred between users connected to one or more of these pages through
	a dynamically updating web interface.

		\subsection[Motiviation]{Project Motivation}
		The primary purpose of this plugin is to enable quick and easy 
		communication between Jenkins users through the provision of an 
		integrated chat client as an augmentation to the Jenkins web interface.
		While other chat clients exist in abundance, the Jenkins chat plugin
		allows for easy integration into a familiar environment with minimal
		installation difficulties.

		\subsection[Goals]{Project Goals}
		The main goal of this project is to provide an intuitive, feature-rich
		chat client within the Jenkins web interface.  By integrating a chat client
		into the Jenkins interface, we aim to remove the need for extraneous chat 
		applications, thus providing a more seamless workflow.  Additionally,
		eliminating the need for extra applications minimizes dependencies on
		additional exterior applications, decreasing both maintenance and
		integration costs for users.

		In an attempt to enhance user satisfaction, we built the application with
		a variety of features, including:

		\begin{itemize}
			\item Peer-to-peer communication between users via a familiar chat
			interface.
			\item Group chat support to allow users to communicate with multiple
			team members simultaneously.
			\item Identification system that allows for user-specified names by
			both system administrators (via an admin file) and chat clients
			(via a user interface in the chat client).
			\item Persistence of chat logs to allow for users and management to 
			cross-reference past communications.
		\end{itemize}
		
		Our hope is that these features will serve to facilitate efficient
		communication between users and provide near-complete coverage of desired
		functionality.


	\section[Functions and Features]{Project Functions and Features}
	The overall architecture of the Jenkins chat plugin follows a traditional
	\href{http://en.wikipedia.org/wiki/Client\%E2\%80\%93server\textunderscore model}{client-server model} 
	where clients to the chat plugin send messages to
	a back-end server and this back-end server forwards incoming messages 
	based on provided recipient information.  The implementation for the server 
	component of the model (and the implementation for supporting back-end
	functionality) is programmed in Java while the client component is
	programmed with a combinaion of Javascript and HTML.  These two components
	communicate using websocket technology (provided by the 
	\href{https://github.com/mrniko/netty-socketio/releases}{Netty-SocketIO} library), 
	which allows for seamless and asynchronous client-server interaction.
	The plugin extends Jenkins to closely integrate the start-up and shut-down
	processes of the server component with the Jenkins server and to provide
	server access to clients through the Jenkins web interface.

		\subsection[Feature List]{Concise Functions and Features List}
		The core of the Jenkins chat plugin consists of the types used to
		implement the server and client components of the client-server model.
		For each of these primary components, the listing below describes the
		major functionality encapsulated by that primary component and provides
		a visual aid for its role in the project through a 
		\href{http://www.csci.csusb.edu/dick/samples/uml0.html}{UML diagram}.

		\subsection[Basic Functions]{Overview of A Few Basic Functions}
		The core of the Jenkins chat plugin consists of the types used to
		implement the server and client components of the client-server model.
		For each of these primary components, the listing below describes the
		major functionality encapsulated by that primary component and provides
		a visual aid for its role in the project through a 
		\href{http://www.csci.csusb.edu/dick/samples/uml0.html}{UML diagram}.

		\subsection[Advanced Functions]{Overview of Advanced Features}
		The core of the Jenkins chat plugin consists of the types used to
		implement the server and client components of the client-server model.
		For each of these primary components, the listing below describes the
		major functionality encapsulated by that primary component and provides
		a visual aid for its role in the project through a 
		\href{http://www.csci.csusb.edu/dick/samples/uml0.html}{UML diagram}.

	\section[Project Implementation]{Project Implementation and Design}
	The core of the Jenkins chat plugin consists of the types used to
	implement the server and client components of the client-server model.
	For each of these primary components, the listing below describes the
	major functionality encapsulated by that primary component and provides
	a visual aid for its role in the project through a 
	\href{http://www.csci.csusb.edu/dick/samples/uml0.html}{UML diagram}.

		\subsection[Database Design]{Data and Database Design}
		The core of the Jenkins chat plugin consists of the types used to
		implement the server and client components of the client-server model.
		For each of these primary components, the listing below describes the
		major functionality encapsulated by that primary component and provides
		a visual aid for its role in the project through a 
		\href{http://www.csci.csusb.edu/dick/samples/uml0.html}{UML diagram}.

		\subsection[Server Implementation]{Implementation Details and Design}
		The core of the Jenkins chat plugin consists of the types used to
		implement the server and client components of the client-server model.
		For each of these primary components, the listing below describes the
		major functionality encapsulated by that primary component and provides
		a visual aid for its role in the project through a 
		\href{http://www.csci.csusb.edu/dick/samples/uml0.html}{UML diagram}.

		\subsection[Implementation Schedule]{Project Implementation Schedule and Tasks}
		The core of the Jenkins chat plugin consists of the types used to
		implement the server and client components of the client-server model.
		For each of these primary components, the listing below describes the
		major functionality encapsulated by that primary component and provides
		a visual aid for its role in the project through a 
		\href{http://www.csci.csusb.edu/dick/samples/uml0.html}{UML diagram}.

			\subsubsection[Task Delegation]{Delegation of Project Tasks}
			The core of the Jenkins chat plugin consists of the types used to
			implement the server and client components of the client-server model.
			For each of these primary components, the listing below describes the
			major functionality encapsulated by that primary component and provides
			a visual aid for its role in the project through a 
			\href{http://www.csci.csusb.edu/dick/samples/uml0.html}{UML diagram}.

			\subsubsection[Challenges]{Technical Challenges}
			The core of the Jenkins chat plugin consists of the types used to
			implement the server and client components of the client-server model.
			For each of these primary components, the listing below describes the
			major functionality encapsulated by that primary component and provides
			a visual aid for its role in the project through a 
			\href{http://www.csci.csusb.edu/dick/samples/uml0.html}{UML diagram}.


	\section[Appendix]{Miscellaneous Project Information}
	The core of the Jenkins chat plugin consists of the types used to
	implement the server and client components of the client-server model.
	For each of these primary components, the listing below describes the
	major functionality encapsulated by that primary component and provides
	a visual aid for its role in the project through a 
	\href{http://www.csci.csusb.edu/dick/samples/uml0.html}{UML diagram}.

		\subsection[Dependencies]{Sources and Dependencies}
		Installation for the chat plugin has the following dependencies, which
		must be installed before the plugin itself can be installed:

		\begin{description}
			\item[\href{http://www.oracle.com/technetwork/java/javase/downloads/jdk7-downloads-1880260.html}{Java Development Kit}] (v1.7+) \hfill \\
			A version of the standard Java development kit at least as recent 
			as version 1.7 is needed primarily to import extended networking 
			functionality only in later JDK versions.

			\item[\href{http://jenkins-ci.org/}{Jenkins}] (v1.509.3+) \hfill \\
			A version of Jenkins more recent than version 1.509.3 is needed
			to support the chat server plugin (older versions may work, but they
			have not been tested).

			\item[\href{https://github.com/mrniko/netty-socketio/releases}{Netty-SocketIO}] (v1.5.2+) \hfill \\ 
			An open source Java library that facilitates network communication 
			via websockets in Java.  This library is used by the chat plugin
			to aid in the transferral of mesages to and from chat clients.
		\end{description}

		After these dependencies are installed, installing the chat plugin itself
		is relatively straightforward.  The steps for performing this installation
		are as follows:

		\begin{enumerate}
			\item Navigate to the \textbf{plugin} subdirectory from the chat
			plugin source base directory.

			\item Run the script within this directory by the name of 
			\textbf{export.sh}.
		\end{enumerate}

		After performing the second step, a sequence of installations should
		follow.  After all these subsequent installations are complete, the
		plugin should be fully installed and accessible through Jenkins.

		\subsection[Other Information]{Other Relevant Information}
		TODO

\end{document}
